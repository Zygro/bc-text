\chapter{Dokumentácia}
\label{kap:doc}
Django servery sledujú návrhový vzor Model-View-Controller (MVC), kde model je tabuľka
databázy a view je spôsob zobrazenia (napríklad vo forme HTML stránky). Pre pochopenie
funkcionality je dôležité vedieť aké modely a pohľady používame a ako medzi sebou interagujú,
preto sa s nimi oboznámime.
\section{Modely}
\label{modely}
Tu sa oboznámime s dôležitými modelmi, teda tabuľkami databázy.
Pri každom modeli spomenieme jeho názov v implementácii a slovenský ekvivalent tohto názvu.
\subsection{Lesson - úloha}
\label{doc:lesson}
Úlohy majú tieto polia:
\begin{itemize}
\itemsep0em
\item \textbf{name}: meno úlohy ktoré sa zobrazuje používateľovi.
\item \textbf{problem}: znenie úlohy.
\item \textbf{pub\_date}: dátum publikácie úlohy.
\item \textbf{number}: úroveň úlohy, používateľ musí mať vyriešené všetky povinné úlohy s
  na nižších úrovniach aby mal k tejto úlohe prístup.
\item \textbf{optional}: voliteľnosť úlohy. Povinné aj voliteľné úlohy sa odomykajú rovnako, voliteľné
   ale neodomykajú neskoršie úlohy.
\item \textbf{inputs}: vstupy na ktorých prebieha testovanie úlohy.
\item \textbf{correct\_solution}: správne riešenie, oproti ktorému sa budú testovať používateľské
   riešenia.
\end{itemize}

Pridávanie úloh je jednoduché - stačí vyplniť polia v administrátorskom prostredí
a uložiť. Na každej úrovni ale musí byť práve jedna povinná úloha, preto po upravení
alebo pridaní novej povinnej úlohy sa nasledovným spôsobom upraví poradie
(funkcia sa zavolá signálom post_save, ktorý sa vykoná po uložení úlohy a nerobí
rozdiel medzi pridávaním novej a upravovaním starej):

Najprv sa všetky povinné úlohy posunú tak, aby neboli žiadne medzery v poradí.
Potom všetky povinné úlohy čo majú úroveň väčšiu alebo rovnú úrovni upravovanej/pridanej
úlohy sú posunuté o jednu úroveň vyššie (upravovaná/pridaná úloha nie je posunutá).
Potom sú všetky úlohy znovu posunuté tak aby neboli medzery (pre prípad, že administrátor
urobil chybu a pridal úlohu s číslom väčším ako má byť najvyššia úroveň, napríklad
ak je 6 povinných úloh a novej úlohe zadá úroveň 10).

Na úlohy máme tri rôzne pohľady (views): Jeden ktorý vráti všetky úrovne, voliteľnosti a názvy úloh ku ktorým má
používateľ prístup, druhý meno, znenie a úroveň s voliteľnosťou jednej úlohy a tretí
iba vstupy alebo skript na stiahnutie (ak používame testovač \ref{testovac1} alebo \ref{testovac2})

\subsection{Submit - riešenie}
Riešenie ktoré používateľ pošle na server na otestovanie.

Polia:
\begin{itemize}
\item \textbf{user}: používateľ ktorý riešenie poslal
\item \textbf{lesson}: úloha ktorú používateľ rieši týmto riešením
\item \textbf{submittedFile}: súbor s riešením, ktoré bude odovzdané testovaču
\item \textbf{result}: výsledok - pravda ak bolo riešenie správne, nepravda ak nebolo
\end{itemize}

Hneď ako server dostane nové riešenie, spustí testovač a jeho odpoveď uloží v poli
\textbf{result}. Ak nie je správne, riešenie uložíme a pokračujeme v činnosti ako pred jeho poslaním.
Ak testovač ale vyhodnotí riešenie ako správne, pred jeho uložením zvýši používateľovu
úroveň o 1 (ak je úloha povinná) a označí úlohu ako vyriešenú.

\subsection{Comment - komentár}
Komentár od používateľa na úlohu.
Polia:
\begin{itemize}
\item \textbf{user}: používateľ ktorý vytvoril komentár
\item \textbf{lesson}: úloha ku ktorej bol komentár pridaný
\item \textbf{text}: text komentára
\item \textbf{date}: čas vytvorenia komentára, potrebný na zoraďovanie pri zobrazení
\end{itemize}

Pri komentároch sa bude zobrazovať aj úroveň autora a to, či úlohu vyriešil.
Zatiaľ sú všetky komentáre prístupné všetkým používateľom (ak majú prístup k ich úlohe)

Je možné že rozdelíme diskusie pre tých čo úlohu vyriešili a tých čo ešte nie
alebo povolíme diskusiu iba pre úspešných riešiteľov, aby žiaden používateľ nemohol
v diskusii prezradiť správne riešenie ostatným používateľom ktorí ešte úlohu nevyriešili.

\subsection{Hint - rada}
Rada k úlohe, pre prípad že ju používateľ nevie riešiť.

Polia:
\begin{itemize}
\item \textbf{lesson}: úloha ku ktorej je táto rada
\item \textbf{number}: poradové číslo rady
\item \textbf{text}: text rady
\end{itemize}
