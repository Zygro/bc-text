\chapter{Server}

\label{kap:server}

V tejto kapitole sa oboznámime s vnútorným fungovaním servra, používaných bezpečnostných protokoloch
a možnosťami poskytovanými administrátorovi.

\section{Použité technológie}

\subsection{Django 1.9}
Django je open source framework nad programovacím jazykom Python ktorý uľahčuje vývoj a údržbu webových aplikácií.
Poskytuje funkcionalitu servra na nízkych úrovňach (správa databázy, HTTP komunikácia a iné), čo umožňuje vývojárovi sa sústrediť na tvorbu samotného obsahu.
Django takisto ponúka jednoducho implementovateľné možnosti ochrany pred niektorými najbežnejšími útokmi ako napríklad XSS, SQL injection alebo CSRF.

Server používa framework Django 1.9 pre jednoduchosť vývoja spravovania databáz, autentifikácie uživateľov a prostredia pre administrátora.
\subsection{REST framework}
Django REST framework je v projekte využitý na jedoduché a prehľadné zobrazovanie obsahu databázy, odpovedí na HTTP požiadavky a na implementáciu
token autorizácie. Umožňuje priamo v prehliadači posielať a zobraziť odpovede na požiadavky, čo zjednodušuje debugovací proces. Taktiež uľahčujú aj implementáciu externej OAuth
autorizácie o ktorej si povieme neskôr

\section{Testovač}
\section{Bezpečnosť}
\subsection{Autorizácia}
Portál používa token autentifikáciu ktorý funguje nasledovne:

Uživateľ pri prihlasovaní pošle prihlasovacie meno a heslo (tento krok je iný pri prihlasovaní cez externú doménu) a ako odpoveď
dostane náhodne vygenerovaný reťazec o dĺžke približne 40 znakov. Potom všetky požiadavky na uživateľove dáta
 musia v hlavičke uviesť tento token, inak budú zamietnuté.
Server si pamätá uživateľov token až kým sa použivateľ neodhlási alebo mu nevyprší platnosť po dlhšej neaktivite.

Vvšetky požiadavky musia obsahovať token, čo pomáha uživateľa chrániť pred CSRF útokom. Narozdiel od cookies,
prehliadač nepridáva token do hlavičky automaticky a teda väčšina CSRF útokov bude zamietnutých.

\subsection{Prihlasovanie cez externé portály}
Portál má viacero možností prihlasovania sa cez externé portály. Použijeme prihlasovaciu schému OAuth 2.0 aby komunikácia s externým
portálom prebiehala iba pri prihlasovaní po ktorom komunikácia bude prebiehať iba medzi klientom a servrom.

Prihlasovanie cez OAuth 2.0 prebieha nasledovne:
\begin{itemize}
\item{Použivateľ sa prihlási na externý portál a potvrdí našej aplikácii vyžiadané oprávnenia}
\item{Od externého portálu získa prístupový token na ten externý portál}
\item{Tento token pošle nášmu servru}
\item{Náš server pošle token externému portálu spolu s tajným ID našej aplikácie aby externý portál vedel že požiadavka prišla od našej aplikácie}
\item{Ak externý portál potvrdí požiadavku, náš server vráti použivateľovi náš token a ďalej komunikujú iba s týmto novým tokenom.}
\end{itemize}

Pomocou tejto schémy sa bude dať prihlásiť cez Facebook, Google+ a GitHub.

\subsection{Protokoly}
Aby sme chránili použivateľové dáta a token, všetky požiadavky obsahujúce osobné dáta používajú šifrovaný protokol HTTPS. Keďže HTTPS je pomalší na spracovanie,
používame ho iba na potrebné požiadavky. Preto verejne známe informácie ako napríklad zadania úloh alebo ich zoznam nemusia byť chránené a používajú rýchlejší HTTP.
\section{Administrácia}