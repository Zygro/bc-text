\chapter{Návrh servra}

\label{kap:server}

V tejto kapitole sa oboznámime s návrhom a špecifikáciami návrhu vnútorného fungovania servera,
bezpečnostných protokoloch a možnosťami poskytovanými administrátorovi.

Je možné, že niektoré z týchto funkcií servera nebudú implementované v prvej pracovnej verzii,
ale budú prítomné vo finálnom produkte.

\section{Použité technológie}

\subsection{Django 1.9}
Django je open source framework nad programovacím jazykom Python, ktorý uľahčuje vývoj a údržbu webových aplikácií.
Poskytuje funkcionalitu servera na nízkych úrovniach (správa databázy, HTTP komunikácia a iné), čo umožňuje vývojárovi sa sústrediť na tvorbu samotného obsahu.
Django takisto ponúka jednoducho implementovateľné možnosti ochrany pred niektorými najbežnejšími útokmi, ako napríklad XSS, SQL injection alebo CSRF.
\cite{Django}

Server používa framework Django 1.9 pre jednoduchosť vývoja spravovania databáz, autentifikácie používateľov a prostredia pre administrátora.
\subsection{REST framework}
Django REST framework je v projekte využitý na jednoduché a prehľadné zobrazovanie obsahu databázy, odpovedí na HTTP požiadavky a na implementáciu
token autorizácie. Umožňuje priamo v prehliadači posielať a zobraziť odpovede na požiadavky, čo zjednodušuje debugovací proces.
Uľahčuje aj implementáciu externej OAuth autorizácie (\ref{oAuth})

Takisto obsahuje implementáciu pre\_save a post\_save signálov ktoré používame pri
modeloch ktorých pridanie alebo zmeny priamo menia obsah iných modelov \cite{rest}
(napríklad pri automatickom zarovnávaní úloh (\ref{doc:lesson})).
\section{Testovač}
\label{testovace}
Testovač je najdôležitejšia časť portálu, pretože overí správnosť používateľovho riešenia. Sú tri možné spôsoby testovania:
\subsection{Stiahnutie vstupov a poslanie výstupov}
\label{testovac1}
Toto je najjednoduchší spôsob implementácie testovača. Používateľ si z nášho portálu si stiahne testovacie vstupy, na ktorých spustí
svoj program, výstup zapíše do súbora a súbor pošle na server, ktorý skontroluje správnosť odpovede.

 Veľká výhoda tohto prístupu je jeho jednoduchosť, ale má značnú nevýhodu: nezaručuje že používateľ napísal optimálny program.
Účelom portálu je učiť používateľov dynamické programovanie, ale tento testovač umožňuje správne vyriešiť úlohu aj hrubou silou.

Tento problém by sa dal obísť používaním úloh, ktoré majú mimoriadne časovo zložité riešenie hrubou silou (napríklad \(O(2^n)\)), ktoré by na väčších vstupoch
vyžadovali niekoľko dní počítania.
Toto riešenie je samozrejme nežiadúce keďže veľmi obmedzuje výber úloh. Preto bude toto testovanie používané iba počas procesu vývoja a nie vo finálnom produkte.

\subsection{Testovanie skriptom u používateľa}

\label{testovac2}
Druhý možný spôsob testovania je podobný predošlému s jednou zmenu: používateľ si miesto vstupov stiahne testovací skript, ktorý spustí používateľom zadaný program
na vstupoch, ktoré skript stiahne z nášho servera. Potom odošle výstupy a tie sú porovnané so správnymi odpoveďami.

Toto riešenie nám umožňuje merať čas potrebný na beh programu, čo je značné vylepšenie oproti predošlému riešeniu, pretože ak používateľ nijak nezasiahne
do skriptu, vieme zistiť či sa mu podarilo úlohu riešiť časovo optimálnym algoritmom.

Používateľ ale môže skript viacerými spôsobmi napadnúť, napríklad manipulovať s časovým obmedzením alebo získať vstupy, vyriešiť ich iným, pomalším algoritmom a
vytvoriť program ktorý pre každý vstup vypíše správny výstup zo súboru alebo priamo zapísaný v zdrojovom kóde.

Tieto chyby sa dajú vyriešiť zabezpečením skriptu proti útokom, ale žiadna ochrana nie je stopercentná. Mohli by sme teda napríklad neakceptovať výsledky, ktorých podozrivo vyzerajúce
časy (t.j. časy, ktoré sa nesprávajú podľa krivky, časy dlhšie ako limit, ktoré ale skript sa nezastavil atď.). Toto riešenie ale nie je akceptovateľné, pretože by mohlo stále prepustiť
falšované výsledky alebo odmietnuť správne riešenie.

Na ochranu vstupov sa dá použiť procedurálna generácia alebo náhodné vyberanie z väčšej množiny vstupov, ale tie pridávajú prácu administrátorovi a komplikujú pridávanie nových úloh.

\subsection{Testovanie na serveri}
Tretí spôsob testovania prebieha na našom serveri. Používateľ pošle zdrojový kód svojho riešenia, ten na serveri skompilujeme a vyhodnotíme s časovým obmedzením behu.

Veľkou výhodou je, že používateľ nemá prístup k testovaným vstupom a správnym výstupom,
takže úloha nebude vyhodnotená za správne vyriešenú pri neoptimálnom programe kvôli vyprchaniu
časového limitu (závisí od dĺžky vstupu a časového limitu, ktoré zadáva administrátor).
Rovnako je oveľa ťažšie napadnúť a upraviť testovač.

Nevýhodou testovača je napríklad zložitá implementácia, pretože musí vedieť kompilovať čo
najviac programovacích jazykov a zároveň musíme dávať pozor aby poslaný program nijako
nenarušoval bezpečnosť servera (nečítal z pamäte ktorú nealokoval, nespúšťal žiadne iné programy,
nepoužíval niektoré systémové volania a podobne). Viac o týchto problémoch
rozpráva zdroj \cite{misof}. Ďalšou nevýhodou sú nároky na výpočtový čas servera:
keďže proces testovania spúšťa program na serveri, môže server spomaliť.

Nakoľko pri tomto testovači je najťažšie mať správne výsledky v časovom limite a
nemať optimálne riešenie, vo finálnom produkte bude implementovaný testovač s týmto prístupom,
napríklad implementácia Mareša-Blackhama \cite{mares}. Ak ale sa zvýši
popularita nášho portálu a vznikne problém s rýchlosťou testovania, bude možno potreba zváženia
vylepšenia hardvéru servera alebo implementácie iného testovača.

\section{Bezpečnosť}
\subsection{Autorizácia}
\label{authorization}
Portál bude používať token autentifikáciu ktorá z pohľadu používateľa vyzerá nasledovne:

Používateľ pri prihlasovaní pošle prihlasovacie meno a heslo (tento krok je iný pri prihlasovaní cez externú doménu) a ako odpoveď
dostane náhodne vygenerovaný reťazec o dĺžke približne 40 znakov. Potom všetky požiadavky, pre ktoré server overuje totožnosť používateľa
musia v hlavičke uviesť tento token. Ak tieto požiadavky uvedú neplatný alebo žiadny token, budú zamietnuté.
Server si pamätá používateľov token až kým sa používateľ neodhlási alebo mu nevyprší platnosť po dlhšej neaktivite.

Všetky požiadavky musia obsahovať token (buď používateľov alebo CSRF token),
čo pomáha používateľa chrániť pred CSRF útokom.

Jednou zo špecifikácií je prihlasovanie cez externé portály, preto budeme v neskorších verziách implementovať
aj prihlasovanie pomocou schémy OAuth 2.0 (\ref{oAuth})
\section{Administrácia a zbieranie dát}
Administrátorovi poskytneme prostredie, v ktorom bude môcť prezerať, pridávať,
upravovať ale aj mazať všetky modely ku ktorým má prístup. Oprávnenie
používať toto prostredie bude mať samozrejme iba administrátor.
\subsection{Pridávanie a hodnotenie úloh}
Pridávanie a spravovanie úloh je hlavnou povinnosťou administrátora. Administrátor
okrem zadávania úloh musí poskytnúť aj vzorové riešenie, prípadne nejaké rady k
riešeniu úlohy a zaradiť ju podľa obtiažnosti.

\subsection{Zbieranie dát o používateľoch}
\label{zbieraniedata}
Náš portál bude okrem vyučovania dynamického programovania schopný aj zbierať údaje o
schopnostiach používateľov a ich zlepšovaní sa postupom cez naše vyučovanie.
Server bude zbierať viacero možných údajov o používateľskej aktivite:
\begin{itemize}
\item Počet navštívených rôznych stránok na našom portáli
\item Časy návštev počas riešenia úloh (aby sme zistili, ako často a intenzívne používateľ používa naše návody na pomoc s riešením)
\item Čas od prečítania úlohy po jej správne vyriešenie (nie veľmi dôležitý, pretože neberie do úvahy prestávky pri riešení)
\item Počet neúspešných pokusov o riešenie
\item Rozdiel priemerného a používateľovho hodnotenia úlohy (až keď bude dostatočne veľa hodnotení aby tento údaj niečo znamenal)
\end{itemize}

Z týchto údajov budeme vedieť zistiť, ako efektívne je vyučovanie
(napríklad, žiak sa pravdepodobne zlepšil ak prvé úlohy označoval ťažšie ako priemer a neskoršie úlohy považoval za ľahšie).
Spätnú väzbu môže administrátor použiť na zlepšenie kvality výučby
a prípadne môžme neskôr implementovať automatické odporúčanie úloh podľa používateľových schopností.

\subsection{Zbieranie dát o úlohách}
Pre kvalitnejší výučbový proces je nutné vedieť, ktoré úlohy sú populárne a koľko
sa na nich žiaci naučia. Preto z používateľskej aktivity budeme interpretovať tieto údaje:
\begin{itemize}
\item Priemerné hodnotenie zložitosti a zaujímavosti úlohy
\item Priemerný čas od zadania do vyriešenia úlohy
\item Priemerný počet nesprávnych riešení pred správnym
\item Priemerný počet použitých rád
\end{itemize}

Tieto dáta bude môcť vidieť iba administrátor a na ich základe by mal meniť úlohy
tak, aby používatelia boli s nimi čo najspokojnejší.

Administrátor by sa preto mal snažiť, aby každá úloha mala ideálne používateľské hodnotenie (\ref{rating})
Je najlepšie, aby každá úloha mala čo najvyššie hodnotenie zaujímavosti,
nakoľko zaujímavé úlohy lepšie motivujú používateľov.

S hodnotením zložitosti je to ale inak: žiadna úloha by nemala byť príliš ľahká (aby bola pre používateľa výzvou) ani
príliš ťažká (aby ju používateľ zvládol), preto ideálny priemer hodnotenia zložitosti je 3.

Server bude poskytovať administrátorovi možnosť posúvať úlohy vyššie a nižšie v poradí,
v ktorom ich používatelia riešia. Takisto bude označovať úlohy, ktoré majú príliš vysokú alebo
príliš nízku obtiažnosť vzhľadom na úroveň riešiteľa a navrhovať nové miesto v poradí, kam ich zaradiť.

Podobne bude označovať úlohy, ktoré sú považované za najmenej zaujímavé a bude na administrátorovi
aby posúdil, či daná nezaujímavá úloha je dôležitá pre proces výučby, alebo či ju možno zmeniť alebo odstrániť.

\subsection{Oprávnenia používateľa a administrátora}
Bežní používatelia majú prístup iba k učebným materiálom, odomknutým úlohám
(vyriešeným aj nevyriešeným)
 a svojim riešeniam. Ďalšie úlohy sa používateľovi odomknú,
iba ak správne vyrieši všetky prerekvizitové úlohy. Všetky dáta zozbierané serverom (\ref{zbieraniedata})
budú od používateľa skryté, pretože by mohli spôsobiť frustráciu u pomalšie sa učiacich používateľov.

Administrátor bude mať prístup ku všetkým úlohám, riešeniam, zozbieraným dátam všetkých používateľov
a ich priemerom. Bude mať aj možnosť manuálne meniť niektoré hodnotenia a odomykať alebo zamykať používateľom úlohy.

Túto možnosť obchádzať pravidlá administrátorovi dávame pre prípad, že vznikne chyba a
používateľovi sa napríklad odomkne úloha ku ktorej nemal mať prístup alebo neodomkne taká,
ktorej všetky prerekvizity splnil. Používateľ v prípade takejto chyby bude môcť kontaktovať
administrátora, ktorý preverí, či naozaj nastala chyba a bude ju môcť napraviť bez debugovania servera.
