\chapter{Dokumentácia stránok}
\lstdefinelanguage{JavaScript}{
  keywords={typeof, new, true, false, catch, function, return, null, catch, switch, var, if, in, while, do, else, case, break},
  keywordstyle=\color{blue}\bfseries,
  ndkeywords={class, export, boolean, throw, implements, import, this},
  ndkeywordstyle=\color{purple}\bfseries,
  identifierstyle=\color{black},
  sensitive=false,
  comment=[l]{//},
  morecomment=[s]{/*}{*/},
  commentstyle=\color{darkgray}\ttfamily,
  stringstyle=\color{red}\ttfamily,
  morestring=[b]',
  morestring=[b]"
}
\lstset{
   language=JavaScript,
   backgroundcolor=\color{lightgray},
   extendedchars=true,
   basicstyle=\small\ttfamily,
   showstringspaces=false,
   showspaces=false,
   numbers=left,
   numberstyle=\footnotesize,
   numbersep=9pt,
   tabsize=2,
   breaklines=true,
   postbreak=\raisebox{0ex}[0ex][0ex]{\ensuremath{\color{red}\hookrightarrow\space}},
   showtabs=false,
   captionpos=b
}

\label{kap:frontImpl}
V tejto kapitole si priblížime implementáciu prednej časti portálu, s ktorou
budú používatelia interagovať. Povieme si komunikácii so servrom ale aj o
implementácii vizualizácie rekurzívnych výpočtov.
\section{Stránky a ich adresy}
V prvej pracovnej verzii sú implementované tieto stránky (uvedené s URL adresami)
\begin{itemize}
  \item registrácia - \textit{/user/register/}
  \item prihlasovanie - \textit{/user/login/}
  \item zoznam úloh - \textit{/lessons/}
  \item detail úlohy - \textit{/lesson/id/} (id je unikátne identifikačné číslo úlohy)
  \item vizualizácia - \textit{/visualisation/}
  \item administrátorské prostredie - \textit{/admin/}
\end{itemize}

Administrátorské prostredie je predvolené prostredie Django frameworku bez modifikácií a je
samozrejme prístupné iba prihláseným administrátorom.
\subsection{Komunikácia so servrom}
Na komunikáciu so servrom používame hlavne knižnicu jQuery, kvôli jej jednoduchému
používaniu a podpore pridania autorizačnej hlavičky. Typická požiadavka vyzerá takto:

\begin{lstlisting}[ language = Javascript, title = Požiadavka na získanie zadania jednej úlohy]
$.ajax({
   url: "../../api/lesson/"+id+"/",
   type: "GET",
   headers: {'Authorization':'Token '+localStorage.getItem('token')},
   success: function(data) {
     title = document.getElementById("title")
     title.innerHTML = "Lesson "+data['name']
     problem = document.getElementById("problem")
     problem.innerHTML = "Lesson "+data['problem']
   },
   error: function(data) {
     if (data.status==401) {
       window.location.replace("../../user/login/")
     }
   }
});
\end{lstlisting}
(v tomto príklade je požiadavka trochu zmenená oproti implementácii, aby bola v texte
prehľadnejšia)


Teraz si vysvetlíme niektoré časti tohto kódu:

V druhom riadku používame premennú id, ktorá
