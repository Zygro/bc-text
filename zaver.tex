\chapter*{Záver}
\addcontentsline{toc}{chapter}{Záver}
Implementovali sme základné prostredie pre výučbu a poskytli administrátorom
nástroje na postupné zlepšovanie výučby. Dávali prednosť funkcionalite,
čím sme docielili, že už prvá pracovná verzia projektu obsahuje takmer všetky funkcie popísané
v špecifikáciách. Používateľské prostredie je preto veľmi jednoduché a slúži hlavne na demonštráciu
možností portálu. Táto verzia je solídny začiatok pre budúce zmeny dizajnu a učebného
procesu.

Vizualizácia môže byť použitá vo finálnej verzii už ako je. Jediné zmeny ktoré
je možno potreba implementovať sú kozmetické, aby lepšie vyhovovala celkovému dizajnu.

Najväčší priestor na zlepšenie je v testovači, ktorý v momentálnej podobe testuje riešenia
bez časového obmedzenia. Preto nie je vhodný na výučbu dynamického programovania, ktorého
hlavným cieľom je rýchlejší výpočet.
Je preto potreba tento testovač čím skôr nahradiť testovaním na servri.

Ostatné oblasti projektu môžu byť zlepšené, ale je možné ich použiť vo finálnej
verzii aj v ich momentálnom stave. Po implementácii testovania na servri by sme
sa mali koncentrovať na zlepšenie prostredia pre používateľov pre ich väčšiu spokojnosť.

Pri momentálnych špecifikáciach je výučba veľmi individuálna, jediná interakcia
medzi rôznymi používateľmi je vo forme diskusie. Pri súčasnom trende rastúceho využívania
sociálnych sietí, stojí za uváženie priadnie sociálnych aspektov k našej výučbe alebo
integrácia so sociálnymi sieťami.

Spoliehame sa na sebamotiváciu používateľov, čo ale môže viesť k nízkej popularite portálu.
Preto v neskorších fázach vývoja odporúčame pridanie požiadavok na motivovanie
a hodnotenie žiakov, napríklad pridanie kompetitívnych prvkov,
ktoré by dovolili používateľom sa porovnávať so svojimi priateľmi.

Nakoniec, pre lepšie výsledky výučby odporučíme spustenie až v neskoršej verzii
s testovaním na servri a vylepšeným používateľským prostredím pre väčšiu spokojnosť používateľa.
