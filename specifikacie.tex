\chapter{Špecifikácie projektu}
Tu si uvedieme špecifické požiadavky na náš portál.

\section{Popis}
Projektom je implementácia jednoduchého internetového portálu slúžiaceho na interaktívnu výučbu
dynamického programovania.

Portál má pozostávať z databázového servra a používateľského prostredia
prístupného prehliadačom.

Okrem výučbovej funkcie, tento portál má implementovať aj zbieranie dát
o výkone používateľov.
\section{Učebný proces}
Výučba je rozdelená do samostatných implementačných úloh, ktoré budú automaticky
testované na servri. Tieto úlohy sú rozdelené do úrovní obtiažnosti.
V každej úrovni je práve jedna povinná a ľubovoľný počet nepovinných úloh.
Používatelia budú postupovať po úrovniach riešením povinných úloh.

Projekt obsahuje aj interaktívnu vizualizáciu rekurzívnych výpočtov,
v ktorej si používatelia budú mocť pozrieť proces výpočtu ľubovoľnej rekurzívnej funkcie.

\section{Prihlasovanie a oprávnenia}
Portál má poskytovať viacero možností autorizácie: klasickú registráciu
na náš server a prihlasovanie cez externé siete ako Facebook.

Kontá v našom portáli sú rozdelené jednoducho na používateľské a administrátorské.
\newpage
Používateľ bude mať možnosť:
\begin{itemize}
  \item Prezerať si úlohy ktoré sú na jeho úrovni alebo nižšej
  \item Posielať riešenia týchto úloh
  \item Hodnotiť úlohy, ktoré správne vyriešil
  \item Nechávať komentáre k úlohám
  \item Prezerať si svoj postup a porovnávať sa s ostatnými používateľmi
  \item Používať s vizualizáciu rekurzívnych výpočtov
\end{itemize}
Administrátor bude mať navyše ešte tieto možnosti:
\begin{itemize}
  \item Upravovať a pridávať úlohy
  \item Prezerať a upravovať výsledky všetkých používateľov
  \item Prezerať dáta pozbierané servrom
\end{itemize}
