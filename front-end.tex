\chapter{Návrh používateľského prostredia}

\label{kap:front}

V tejto kapitole si predstavíme používateľské prostredie a ako bude výučba vyzerať
z pohľadu používateľa.

\section{Proces výučby}
\subsection{Postup úlohami}
Hlavný spôsob výučby poskytovaný naším portálom je zadávanie implementačných úloh,
ktoré používateľ na svojom zariadení vyrieši a pošle nám riešenie, ktorého správnosť
overíme. Ďalšie správanie portálu závisí na správnosti riešenia
(podľa implementácie testovača (\ref{testovace}) sa pod riešením môže rozumieť
samotný kód alebo archivované súbory s výstupmi).

Úlohy sú rozdelené na povinné, nepovinné a výučbové texty, ktoré nie sú testované.
Taktiež sú zoradené do rôznych úrovní.
Na každej úrovni je jedna povinná a ľubovoľný počet nepovinných úloh a textov.
Ak používateľ správne vyrieši povinnú úlohu, jeho úroveň sa zvýši o 1 a
odomknú sa mu nové úlohy.

\subsection{Rady}
Ak si používateľ nevie rady s úlohou, má možnosť požiadať o radu. Každá úloha
môže mať ľubovoľne veľa rád - podľa uváženia administrátora. Používateľ pri začatí riešenia
úlohy nebude mať k dispozícii žiadnu radu, ale môže o ňu kedykoľvek požiadať tlačidlom na
stránke úlohy, ktoré mu sprístupní jednu ďalšiu radu.
Používateľovi sa vždy pri zobrazení úlohy ukážu aj všetky rady ku ktorým má prístup.

\subsection{Hodnotenie úloh}
\label{rating}
Používatelia môžu po správnom vyriešení úlohu ohodnotiť.

Budú mať možnosť (ale nie povinnosť) ohodnotiť jej obtiažnosť a
zaujímavosť na stupnici od 1 do 5. Tieto hodnotenia sa priemerujú a tieto priemery
budú sprístupné iba administrátorovi. Každé hodnotenie je prístupné iba používateľovi ktorý
ho vytvoril a administrátorovi. Každý používateľ môže ohodnotiť úlohu iba raz.

\subsection{Diskusia}
Každá úloha bude mať diskusiu, v ktorej bude možné rozoberať úlohu.
Pri implementačných úlohách ale môžu vzniknúť rôzne problémy a potreba moderácie diskusie.
O týchto problémoch si povieme
viac v kapitole o budúcom vývoji \ref{future:disc}

\section{Prostredie}
Predstavíme jednotlivé stránky ku ktorým bude používateľ mať prístup a aké
funkcie poskytujú.

\subsection{Základné stránky}
Portál obsahuje stránky na registráciu nových používateľov, prihlásenie, zoznam úloh a
používateľov profil. Tieto stránky sú jednoduché a ich funkcia evidentná, preto ich iba spomenieme
bez podrobného popisu.

\subsection{Detail úlohy}
Stránka, ktorá zobrazí detail jednej úlohu: jej názov, zadanie, rady o ktoré používateľ požiadal, hodnotenie jeho riešení
generované testovačom a nasledujúce možnosti:
\begin{itemize}
  \item stiahnuť vstupné dáta alebo testovací skript - poskytnutie tejto možnosti závisí od implementácie testovača (\ref{testovace})
  \item odovzdať riešenie na otestovanie
  \item po správnom vyriešení úlohy ju ohodnotiť (\ref{rating})
  \item zobraziť diskusiu a pridať komentár
  \item požiadať o ďalšiu radu
\end{itemize}

\subsection{Vizualizácia rekurzívnych výpočtov}
Jedna z najťažších častí učenia sa dynamického programovania je pochopenie rekurzívnych funkcií.
Preto náš portál poskytuje možnosť si v prehľadnej forme zobraziť výpočet
rekurzívnej funkcie zadanej používateľom.

Na stránke vizualizácie si používateľ môže napísať svoju funkciu, ktorej stránka vytvorí
a zobrazí strom výpočtu. Každý vrchol tohto stromu reprezentuje jedno zavolanie funkcie.

Vrcholy budú vykresľované postupne, podľa poradia zavolania.
Vykreslené vrcholy ukazujú všetky svoje argumenty. Návratová hodnota zavolania
sa zobrazí hneď, ak funkcia samú seba ďalej nevolá. Ak volanie funkcie zavolá funkciu znova (alebo vrchol volania má deti),
výsledok zobrazíme až keď všetci potomkovia jej vrchola v strome
majú zobrazenú hodnotu. Týmto sa snažíme o to, aby používateľ lepšie videl, ako postupoval výpočet a v ktorom momente výpočtu
boli spočítané ktoré návratové hodnoty.

Vizualizácia vie pracovať aj s memoizáciou, čiže pamätaním si návratových hodnôt pre
argumenty a odpovedanie z pamäte pri každom ďalšom zavolaní funkcie s rovnakými argumentmi.
Pri výpočte s memoizáciou vizualizácia zobrazuje aj zoznam argumentov a ich hodnôt, ktoré už boli vypočítané.
V prípade vykreslenia vrcholu s argumentmi, pre ktoré hodnotu už poznáme, ukážeme aj na vrchol
kde bola hodnota prvý krát vypočítaná.

Aj pre túto stránku si uvedieme možnosti, ktoré ponúka používateľovi:
\begin{itemize}
  \item vybrať si, či výpočet bude používať memoizáciu alebo nie
  \item vyhodnotiť zadanú funkciu a spustiť vizualizáciu jej výpočtu
  \item zastaviť/znova spoustiť vizualizáciu
  \item krokovať vizualizáciu dopredu alebo dozadu (pri krokovaní vizualizáciu
   automaticky zastavíme)
  \item vykresliť celý strom výpočtu alebo začať vizualizáciu odznova
\end{itemize}

Uprednostníme počítanie a vytvorenie stromu pred vykreslením oproti počítaniu počas vykresľovania
kvôli jednoduchšej implementácii počítania, ale aj vykresľovania samotného.
