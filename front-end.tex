\chapter{Výučba}

\label{kap:front}

V tejto kapitole si predstavíme používateľské prostredie a ako bude výučba vyzerať
z pohľadu používateľa.

\section{Proces výučby}
\subsection{Postup úlohami}
Hlavný spôsob výučby poskytovaný naším portálom je zadávanie implementačných úloh,
ktoré používateľ na svojom zariadení vyrieši a pošle nám riešenie ktorého správnosť
overíme a ďalšie správanie portálu zácisí na správnosti riešenia
(podľa implementácie testovača (\ref{testovace}) sa pod riešením môže rozumieť
samotný kód alebo archivované súbory s výstupmi).

Úlohy sú rozdelené na povinné a nepovinné a zoradené do rôznych úrovní.
Na každej úrovni je jedna povinná a ľubovoľný počet nepovinných úloh.
Ak používateľ správne vyrieši povinnú úlohu, jeho úroveň sa zvýši o 1 a
odomknú sa mu nové úlohy. Nepovinné úlohy neodomykajú nové úrovne, ale rátajú
sa používateľovi do jeho skóre (\ref{score}).

\subsection{Rady}
Ak si používateľ nevie rady s úlohou, bude mať možnosť požiadať o radu. Každá úloha
má ľubovoľne veľa rád - podľa uváženia administrátora. Používateľ pri začatí riešenia
úlohy nemá k dispozícii žiadnu radu, ale môže o ňu kedykoľvek požiadať tlačidlom na
stránke úlohy, ktoré mu sprístupní jednu ďalšiu radu.
Potom sa mu vždy pri zobrazení úlohy ukážu aj všetky rady ku ktorým má prístup.

\newpage
\subsection{Bodovanie a rebríček}
\label{score}
Každý používateľ má úroveň a skóre ktoré ukazujú jeho postup učebným procesom.

Úroveň používateľa je jednoducho najvyššia úroveň úloh ktoré sa podarilo používateľovi odomknúť

Skóre je o niečo zložitejšie: všetky správne vyriešené úlohy (povinné aj nepovinné)
pridajú používateľovi do skóre svoju úroveň, teda prvá úloha sa ráta za
1 bod a úloha na vyššej, povedzme piatej úrovni pridá 5 bodov.
\newline
\newline
Používatelia majú aj možnosť pozrieť si svoje poradie v rebríčku a porovnať sa
s ostatnými používateľmi. Sú dva rôzne rebríčky: rebríček úrovní a rebríček skóre.

Každý používateľ vidí v rebríčkoch seba a všetkých ostatných používateľov ktorí
súhlasia so zverejnením svojho skóre (používateľ môže tento súhlas vo svojom profile kedykoľvek zmeniť).

K rebríčkom má prístup aj používateľ ktorý nesúhlasí so zverejnením svojej úrovne a skóre.
Takýto používateľ stále vidí svoje poradie v porovnaní s ostatnými používateľmi tak isto
ako ostatní používatelia. Jediná zmena je, že sa nezobrazuje v rebríčkoch ostatným používateľom.

\subsection{Hodnotenie úloh}
\label{rating}
Používatelia mau možnosť po správnom vyriešení úlohu ohodnotiť.

Budú mať možnosť (ale nie povinnosť) ohodnotiť jej obtiažnosť a
zaujímavosť na stupnici od 1 do 5. Tieto hodnotenia sa budú priemerovať a ich priemery
budú prístupné iba administrátorovi. Každé hodnotenie bude prístupné iba používateľovi ktorý
ho vytvoril a administrátorovi.

\subsection{Diskusia}
Každá úloha má diskusiu, v ktorej je možné rozoberať úlohu.
V momentálnom návrhu je celá diskusia prístupná všetkým používateľom ktorí majú
prístup k úlohe, z čoho ale môže vzniknúť problém prezrádzania správnych riešenie
ostatným používateľom, ktorí ešte úlohu nevyriešili

Ak tento problém vznikne, budeme musieť rozdeliť diskusiu na dve: pre tých používateľov, čo úlohu vyriešili a pre tých čo ešte nie.

Ak budú stále pretrvávať problémy (napríklad používateľ prezradí riešenie skôr ako ho odovzdá na testovanie),
môžme povoliť diskusiu iba pre úspešných riešiteľov.

Toto riešenie ale môže byť v rozpore s účelom diskusie, nakoľko sa môže stať, že používatelia sa po vyriešení úlohy
o ňu prestanú zaujímať. Preto riešenie tohto problému zatiaľ necháme na administrátorovi, aby zmazal všetky diskusné
príspevky s radami, ktoré príliš zjednodušujú úlohu a iným nevhodným obsahom.
\section{Prostredie}
Predstavíme jednotlivé stránky ku ktorým má používateľ prístup a aké
funkcie ktoré poskytujú

\subsection{Základné stránky}
Portál obsahuje stránky na registráciu nových používateľov, prihlásenie, zoznam úloh a
používateľov profil ktoré sú jednoduché a ich funkcia evidentná, preto ich iba spomenieme
bez podrobného popisu.

\subsection{Detail úlohy}
Stránka ktorá zobrazuje jednu úlohu: jej názov, zadanie, rady o ktoré používateľ požiadal, hodnotenie jeho riešení
generované testovačom a nasledujúce možnosti:
\begin{itemize}
  \item stiahnuť vstupné dáta alebo testovací skript - poskytnutie tejto možnosti závisí od implementácie testovača (\ref{testovace})
  \item odovzdať riešenie na otestovanie
  \item po správnom vyriešení úlohy ju ohodnotiť (\ref{rating})
  \item zobraziť diskusiu a pridať komentár
  \item požiadať o ďalšiu radu
\end{itemize}
