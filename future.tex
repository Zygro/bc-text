\chapter{Budúci vývoj}
V tejto záverečnej kapitole si určíme ciele budúceho vývoja.

\section{Bodovanie a rebríček}
\label{score}
Každý používateľ bude mať skóre, ktoré ukazuje jeho postup učebným procesom.

Všetky správne vyriešené úlohy (povinné aj nepovinné)
pridajú používateľovi do skóre svoju úroveň. Napríklad prvá úloha sa ráta za
1 bod a úloha na vyššej, povedzme piatej úrovni pridá 5 bodov.
\newline
\newline
Používatelia budú mať aj možnosť pozrieť si svoje poradie v rebríčku a porovnať sa
s ostatnými používateľmi. Sú dva rôzne rebríčky: rebríček úrovní a rebríček skóre.

Každý používateľ bude môcť vidieť v rebríčkoch seba a všetkých ostatných používateľov ktorí
budú súhlasiť so zverejnením svojho skóre (používateľ bude môcť tento súhlas vo svojom profile kedykoľvek zmeniť).

K rebríčkom bude mať prístup aj používateľ, ktorý nesúhlasí so zverejnením svojej úrovne a skóre.
Takýto používateľ stále bude vidieť svoje poradie v porovnaní s ostatnými používateľmi tak isto
ako ostatní používatelia s tou zmenou, že sa nezobrazí v rebríčkoch ostatným používateľom.

\section{Diskusia}
\label{future:disc}
V momentálnom návrhu je celá diskusia prístupná všetkým používateľom ktorí majú
prístup k úlohe, z čoho ale môže vzniknúť problém prezrádzania správnych riešenie
ostatným používateľom, ktorí ešte úlohu nevyriešili.

Ak tento problém vznikne, budeme musieť rozdeliť diskusiu na dve: pre tých používateľov, čo úlohu vyriešili a pre tých čo ešte nie.

Ak budú stále pretrvávať problémy (napríklad používateľ prezradí riešenie skôr ako ho odovzdá na testovanie),
môžme povoliť diskusiu iba pre úspešných riešiteľov.

Toto riešenie ale môže byť v rozpore s účelom diskusie, nakoľko sa môže stať, že používatelia sa po vyriešení úlohy
o ňu prestanú zaujímať. Preto riešenie tohto problému zatiaľ necháme na administrátorovi, aby zmazal všetky diskusné
príspevky s radami, ktoré príliš zjednodušujú úlohu a iným nevhodným obsahom.

\section{Vizualizácia}
Je možné implementovať aj spoluprácu vizualizácie s úlohami, kde by sa úloha mohla odkazovať
na konkrétny vstup vizualizácie, napríklad formou linku ktorý používateľa presmeruje
na stránku vizualizácie kde bude zadaný potrebný kód.

V momentálnom návrhu ale pre túto funkciu nie je potreba, pretože rovnaký efekt docielime
jednoducho napísaním potrebného kódu, ktorý používateľovi odporučíme skopírovať
a vyhodnotiť vo vizualizácii manuálne. Oba spôsoby docielia rovnaký výsledok a
automatické presmerovanie ušetrí iba málo používateľovho času takže ho považujeme
za nepotrebné.

Ďalšia možnosť pre budúci vývoj je napríklad ukázať iba strom výpočtu bez kódu, čo môžme
využiť ako radu k úlohe alebo úplne nový typ úlohy, kde by používateľ musel
nájsť funkciu, ktorá tento strom generuje.

Vizualizácia výpočtu nám zjavne ponúka veľké množstvo možností, ktoré určite v budúcnosti využijeme
pre zlepšenie výučbového procesu.

\section{Prihlasovanie cez externé portály}
\label{oAuth}
Portál bude mať okrem priamej registrácie na server viacero možností prihlasovania sa cez externé portály.
Použijeme prihlasovaciu schému OAuth 2.0, aby komunikácia s externým
portálom prebiehala iba pri prihlasovaní po ktorom používateľov prehliadač
komunikuje so serverom ako keby sa prihlásil priamo na náš portál bez využitia externej služby.

Prihlasovanie cez OAuth 2.0 prebieha nasledovne \cite{FBlogin}:
\begin{enumerate}
\item{používateľ sa prihlási na externý portál a potvrdí našej aplikácii vyžiadané oprávnenia}
\item{Od externého portálu získa prístupový token na ten externý portál}
\item \label{secret}{Tento token pošle nášmu serveru}
\item{Náš server pošle token externému portálu spolu s tajným ID našej aplikácie, aby externý portál vedel, že požiadavka prišla od našej aplikácie}
\item{Ak externý portál potvrdí požiadavku, náš server vráti používateľovi náš token a
    ďalšia komunikácia prebieha iba s týmto novým tokenom ako pri obyčajnej autentifikácii}
\end{enumerate}

Pomocou tejto schémy sa bude dať prihlásiť cez Facebook, Google+ a GitHub.
\newline
Tajné ID použité v kroku \ref{secret} bude prístupné iba administrátorom.
