\documentclass[11pt,a4paper]{article}
\usepackage{fullpage}       % nastav mensie okraje
\usepackage[utf8]{inputenc} % kodovanie utf8
\usepackage[T1]{fontenc}    % kodovanie utf8
\usepackage[slovak]{babel}  % pouzitie slovenciny
\usepackage{url}            % lepsie zobrazovanie URL
\usepackage{bibentry}       % zobrazovanie celych odkazov v texte
\nobibliography*            % odkazy sa maju dat aj do zoznamu literatury

\begin{document}

\title{1-INF-911 Bakalársky seminár (1)\\
Domáca úloha 1}
\date{Školský rok 2015/16}
\author{Michal Smolík}

\maketitle


\section{Základné údaje o bakalárskej práci}

\begin{description}
% TODO: nizsie vyplnte udaje o vasom skolitelovi a praci
\item[Školiteľ:] Michal Foríšek, PhD. 
\item[Pracovisko školiteľa:] Katedra informatiky, FMFI UK
\item[Názov práce:]Portál pre vyučovanie dynamického programovania

\item[Cieľ práce:]Cieľom tejto práce je vytvoriť nový vzdelávací portál
na vyučovanie dynamického programovania, ktorý sleduje postup žiakov
a automaticky vyhodnocuje ich výsledky ako štatistické dáta. Portál
%má ponúkať alebo ponúka?
má ponúkať vizualizácie výpočtov vo vzorových príkladoch a dynamické
priraďovanie úloh podľa žiakovych schopností.
\end{description}

\section{Kľúčové zdroje}

V mojej bakalárskej práci sa plánujem opierať najmä o nasledujúce
informačné zdroje.

% TODO nizsie vyplte vase zdroje
% pre kazdy v prikaze \bibentry dajte kluc z literatura.bib
% vynechajte volny riadok (koniec odstavca) a potom
% strucne popiste, preco je tento zdroj dolezity pre vasu pracu
\begin{itemize}

\item \bibentry{Django}

Dokumentácia technológie Django ktorú používa server

\item \bibentry{Scoops2015}

Kniha obsahuje rady pre tvorbu serverov Django ktoré budem používať

\item \bibentry{React}

Dokumentácia knižnice React, ktorá je použitá vo webstránke

\item \bibentry{Promise}

Návod na použitie štruktúry Promise, ktorá je používaná pri posielaní požiadavok na server

\item \bibentry{FBlogin}

Návod naimplementáciu prihlasovania cez Facebook

\end{itemize}
% Koniec zoznamu zdrojov

% Zapneme pouzitie suboru literatura.bib
\bibliographystyle{plain}
\bibliography{literatura} 

\end{document}
