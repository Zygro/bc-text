\chapter*{Úvod}
\addcontentsline{toc}{chapter}{Úvod}
Dynamické programovanie je pre veľa začínajúcich programátorov zastrašujúce,
pretože vyžaduje zmenu spôsobu rozmýšľania o probléme.

Preto vytvárame miesto, kde sa žiaci môžu aj zo svojho domova učiť o základných princípoch a
konkrétnych využitiach rekurzívnych výpočtov, memoizácie a dynamického programovania interaktívnym spôsobom.
Úlohou portálu je poskytnúť prostredie, kde sa používatelia môžu od základov naučiť
riešiť úlohy na dynamické programovanie bez účasti učiteľa.

Koncentrujeme sa hlavne na poskytovanie priestoru pre aktívnu výučbu,
pretože programovanie sa najlepšie učí skúšaním. Preto sa výučba na našom portáli zameriava na
zadávanie a automatické testovanie implementačných úloh. Ďalšia z hlavných funkcií portálu je
nteraktívna vizualizácia rekurzívnych výpočtov a memoizácie, ktorej úlohou je prehľadným spôsobom ukázať túto niekedy
náročnú, ale mimoriadne dôležitú časť programovania.

Aby sa zaistila vysoká kvalita vyučovania a používatelia ostali motivovaní, zbierame
dáta o ich postupe výučbou a zároveň im dávame možnosť nechať spätnú väzbu na
každú úlohu ktorú správne vyriešili. Tým zaistíme, že administrátor vie o tom,
ktorá úloha je populárna, ktorá nie a ktoré časti výčbového procesu treba vylepšiť.

Je dôležité spomenúť, že cieľom tejto práce je iba implementácia funkcií portálu, nie výčbového
obsahu. Preto budeme rozprávať iba o návrhu a konkrétnej implementácii portálu,
nie o výučbe ako takej.

Táto práca má štyri hlavné časti, v ktorých si povieme o návrhoch prednej
aj zadnej strany nášho portálu a potom aj o ich konkrétnych implementáciách v prvej pracovnej
verzii portálu.
