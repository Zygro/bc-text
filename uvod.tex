\chapter*{Úvod a ciele práce}
\addcontentsline{toc}{chapter}{Úvod}
Dynamické programovanie je pre veľa začínajúcich programátorov zastrašujúce,
pretože vyžaduje zmenu spôsobu rozmýšľania o probléme. Potrebujú sa naučiť,
ako problém rozdeliť na menšie podproblémy a z ich riešení nájsť aj riešenie
pôvodného problému.

Preto vytvárame miesto, kde sa žiaci môžu aj zo svojho domova učiť o základných princípoch a
konkrétnych využitiach rekurzívnych výpočtov, memoizácie a dynamického programovania interaktívnym spôsobom.
Náš portál je prostredie, kde sa používatelia môžu od základov naučiť
riešiť úlohy z dynamického programovania aj bez účasti učiteľa.

Je dôležité najprv spomenúť, že cieľom tejto práce je iba implementácia funkcií portálu, nie výčbového
obsahu. Preto budeme rozprávať iba o špecifikáciach, návrhu ich splnenia a konkrétnej implementácii portálu,
nie o výučbe ako takej.

Hlavným cieľom práce je teda poskytnutie priestoru pre aktívnu výučbu,
pretože programovanie sa najlepšie učí skúšaním. Preto sa výučba na našom portáli zameriava na
zadávanie a automatické testovanie implementačných úloh. Ďalšia z hlavných funkcií portálu je
nteraktívna vizualizácia rekurzívnych výpočtov a memoizácie, ktorej úlohou je prehľadným spôsobom ukázať túto niekedy
náročnú, ale mimoriadne dôležitú časť programovania.

Ďaľším cieľom je zaistenie kvality výučby. Tento cieľ v tejto práci neovplyvníme
priamo, pretože samotný učebný obsah nie je v rozsahu tejto práce, ale poskytneme
viaceré nástroje, ktorými zjednodušíme jej zlepšovanie.

V prvej kapitole si definujeme špecifikácie a požiadavky tohto projektu.
V druhej a tretej kapitole si opíšeme postupne návrh používateľského prostredia a servra
a v posledných dvoch aj ich implementáciu v prvej pracovnej verzii projektu.
